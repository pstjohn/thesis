\chapter{Spatiotemporal separation of PER and CRY posttranslational regulation}
% \section{Biological motivation}\blindtext
% \subsection{Longdaysin vs KL001}\blindtext
% \subsection{Post-translational regulators CKI and FBXL3}\blindtext
% \subsection{Opposite amplitude effect, same period effect}\blindtext
% \section{Identifiability analysis}\blindtext
% \subsection{remove dependence on both parameterization and model assumptions }\blindtext
% \section{Prediction of Longdaysin vs KL001 mechanisms}\blindtext
% \subsection{explanation using single model}\blindtext


\section{Introduction}
Circadian rhythms are autonomous, near-24 hour oscillations that
coordinate daily changes in physiology and metabolism. Since circadian and
metabolic regulators are tightly integrated, circadian disruptions often
manifest in metabolic disease \cite{Bass2012}. Recent efforts have therefore
sought to gain a mechanistic understanding of these pathways, such that the
metabolic burdens imposed by a 24-hour society might be mitigated.
Post-translational regulators, which play key roles in connecting circadian and
metabolic processes, serve as likely targets for future therapeutics --
demonstrated by the wealth of available circadian-active small molecules
\cite{Chen2013}.

Oscillations in circadian gene transcription are generated through a
time-delayed transcription-translation negative-feedback loop. In mammals,
transcription factors CLOCK and BMAL1 promote transcription of E box-containing
genes {\it Period} ({\it Per}) and {\it Cryptochrome} ({\it Cry}) (Fig. 1A).
PER and CRY protein products form heterodimers to accumulate in the nucleus, in
which PER is stoichiometrically limiting \cite{Lee2001}, and subsequently close
the negative feedback loop by inhibiting CLOCK-BMAL1-promoted gene expression.
While steady-state endpoint assays have shown the possibility of nuclear entry
of CRY without PER \cite{Ye2011, Kume1999, Yagita2002}, experiments from {\it
Per1}\textsuperscript{-/-} {\it Per2}\textsuperscript{-/-} mice demonstrated
that PER proteins are required for the timely nuclear accumulation of CRY
\cite{Lee2001}.  Clearance of nuclear repressors reactivates CLOCK-BMAL1,
allowing the cycle to begin anew \cite{Takahashi2008}. 

Experimental evidence on PER/CRY nuclear entry is seemingly contradictory. For
nuclear localization of the PER and CRY proteins, experiments from {\it
Per1}\textsuperscript{-/-} {\it Per2}\textsuperscript{-/-} mice demonstrate
that PER proteins are required for timely nuclear accumulation of CRY
\cite{Lee2001}.  While other studies have shown the possibility of nuclear
entry of CRY without PER, these results are typically based on steady-state
endpoint assays which do not consider the speed of CRY nuclear entry
\cite{Ye2011, Kume1999, Yagita2002}. We therefore consider the formation of
the PER-CRY heterodimer as a key step in nuclear entry, which is supported by
the fact that to the best of our knowledge all circadian models that consider
both PER and CRY employ this kinetic assumption \cite{Hirota2012, Relogio2011,
Leloup2003, Forger2003, Mirsky2009}.

The stabilities of PER and CRY are tightly regulated: PER proteins are
phosphorylated by the casein kinase I family of proteins (CKI$\delta/\epsilon$),
prompting $\beta$-TrCP-mediated degradation \cite{Reischl2007} and nuclear
import \cite{Takano2004}. The degradation of CRY proteins is separately
regulated by the SCF\textsuperscript{FBXL3} ubiquitin ligase complex
\cite{Busino2007, Godinho2007, Siepka2007}. The activities of both CKI-PER and
FBXL3-CRY may be further coupled to the cell's metabolic state through AMPK
signalling \cite{Lee2013}. These post-translational regulatory mechanisms have a
strong effect on period length: the gain-of-function mutant
CKI$\epsilon^\mathrm{tau}$ leading to hyperphosphorylation of PER
\cite{Gallego2006} and small molecule CKI inhibitors, such as longdaysin
\cite{Hirota2010}, demonstrated that increasing or decreasing CKI-dependent PER
phosphorylation shortens or lengthens the period, respectively. In contrast,
genetic mutations of FBXL3 \cite{Godinho2007, Siepka2007} and KL001, a small
molecule inhibitor of FBXL3-dependent CRY degradation \cite{Hirota2012}, showed
that increased CRY stability leads to longer periods. Since the scale and
complexity of the circadian network complicates an intuitive understanding of
these relationships, mathematical models have played important roles in
understanding how these manipulations affect circadian period \cite{Gallego2006,
Hirota2012, Reischl2007}.

Given that both CKI and FBXL3 pathways regulate the stability of linked
negative factors, it was thought that simultaneous perturbations to both
pathways might lead to non-additive effects: {\it i.e.}, the slowest link would
determine the period.  However, both small molecule \cite{Hirota2012} and
genetic experiments \cite{Maywood2011} have demonstrated the independent period
effects of these two post-translational regulations. A recent clarification of
the canonical clock feedback circuit has shown that dissociated CRY is the
dominant repressor of CLOCK-BMAL1 mediated E box transcription \cite{Ye2011}.
This distinction helps differentiate between the roles of the otherwise similar
PER and CRY proteins, in which the main role of PER in transcriptional
repression is likely regulating the timing of nuclear accumulation of CRY.
Therefore, while previous mathematical models in which PER acts as a direct
repressor have proposed mechanisms for CKI-dependent period lengthening
\cite{Gallego2006, Vanselow2006}, they are likely not suitable for
distinguishing between CKI-PER and FBXL3-CRY mediated period change.

In this study, we used human cells harboring clock gene reporters together with
mathematical modeling to gain insight into the relationship between PER and CRY
post-translational regulation. Consequently, we provide a new mechanism by
which CKI-dependent PER phosphorylation controls the circadian period
separately from the FBXL3-CRY pathway. The resulting detailed understanding of
PER and CRY regulation in the core feedback loop provides a framework on which
to interpret metabolic and pharmacological control of circadian rhythms.

\section{Materials and Methods}
\subsection{Analysis of luminescence profiles}
Raw luminescence data was first separated into a moving baseline and oscillatory
component using a Hodrick-Prescott filter with a smoothing parameter of 1600.
Example trajectory decompositions are shown in Figure S1. Amplitudes (as shown
in Fig. 1C) were determined by taking the standard deviation in the
baseline-subtracted data. Periods were obtained by nonlinear curve fitting, in
which a four parameter (initial amplitude, decay, period and phase) damped
cosine curve was fit to the baseline-subtracted data. Periods were not shown if
the relative amplitude (found by standard deviation) fell below 25\%, since
noise dominated the periodic trajectory. 

\subsection{Cost function}
Models were fit to a cost function of experimental results. {\it Per}, {\it
Cry}, {\it Clock}, and {\it Bmal1} protein and mRNA levels were taken from
\cite{Lee2001}, along with profiles of CRY nuclear localization. For the model
from \cite{Relogio2011}, additional activity profiles on {\it Rev-Erb} and {\it
Ror} were obtained from CircaDB (http://bioinf.itmat.upenn.edu/circa/). To score
a model trajectory, mRNA state variables were scaled independently to minimize
the squared error between model and experiment, since model parameters could be
adjusted to give mRNA profiles arbitrary amplitudes. For protein species, where
stoichiometric interactions are important, a single scaling parameter was used
for all species. Nuclear repressor species, in which only relative measurements
were available, were scaled independently. Full model equations are shown in the SI.

\subsection{Parameter estimation and bootstrap analysis}
Bootstrap parameter estimations were performed as described previously
\cite{St.John2013}, with data from \cite{Lee2001} assumed to have a normally
distributed 10\% relative and 5\% absolute error. Since not all states in the
models were measured, initial guess values for the trajectory and parameter
variables were generated by optimizing the parameter sets first with a genetic
algorithm approach, described in \cite{Mirsky2009}. To help ensure bootstrap
trials remained in a similar stability region of parameter space (and protect
against steady-state solutions), bootstrap parameters were bound between 50\%
and 150\% of their initial value. 

\subsection{Selection of parameters for FBXL3-CRY and CKI-PER mechanisms}
For FBXL3-CRY, parameters that determined the degradation rate of CRY (or CRY
containing complexes) were considered to be the most likely candidates.
Michealis-Menten degradation parameters were omitted from Fig. 3 since
perturbations to such parameters are not easily attributable to changes in
FBXL3 binding affinity. In the model presented in \cite{Leloup2003}, CRY is
degraded through a series of phosphorylation events, and these parameters were
considered as representative of the rate of progression toward ubiquitination
of CRY. The forward phosphorylation rates of CRY and nuclear PER-CRY complex
were therefore also considered.  For CKI-PER, we considered rates that
determined the degradation rate and nuclear import rate of PER.
Michealis-Menten parameters were not included, similar to FBXL3-CRY. With CRY
being the main repressor of E box transcription \cite{Ye2011}, the degradation
rates of PER-CRY complex were not considered as potential mechanisms of CKI. In
the models of \cite{Leloup2003} and \cite{Relogio2011}, the nuclear entry of
PER-CRY requires two independent steps: the formation of the PER-CRY complex
and the subsequent import of the complex.  Therefore, the forward reaction
rates of each of these steps were included.

\subsection{Numerical experiments}
Numerical parameter inhibitions were performed by recalculating the limit cycle
trajectory for each new parameter set to a tolerance of $10^{-8}$, using
computational methods described previously \cite{Wilkins2009}.


\section{Results and Discussion}
\subsection{Longdaysin and KL001 yield opposite effects on the amplitude of
circadian reporter expression}
To gain a more detailed understanding of the roles of CKI-PER and FBXL3-CRY
pathways, we applied small molecule
compounds longdaysin and KL001, which cause stabilization of PER and CRY,
respectively \cite{Hirota2010, Hirota2012} (Fig. 1A). We used {\it Bmal1}- and
{\it Per2-dLuc} as circadian reporters, which represent different loops of the
core clock mechanism and show circadian luminescence rhythms with mutually
opposite phase.  Time-course data on circadian reporter expression under
increasing concentrations of longdaysin and KL001 \cite{Hirota2012} was
analyzed for period and amplitude change (Figs. 1B, 1C, S1). Longdaysin caused
dose-dependent increases in period and detrended amplitude to $\approx 50\%$ of
control values in both {\it Bmal1}- and {\it Per2-dLuc} reporter cells. In
contrast, KL001 induced a simultaneous increase in period and strong reduction
in amplitude. Modulation of the activity of CKI-PER and FBXL3-CRY is therefore
differentiated by an opposite amplitude response.

\subsection{Bootstrap approach reveals main period-determining perturbations}

We next used {\it in silico} modeling to gain mechanistic insight into CKI-PER
and FBXL3-CRY mediated circadian regulation. We previously described the
connection between inhibition of FBXL3-dependent CRY degradation and period
change \cite{Hirota2012}: increasing the stability of nuclear CRY results in
longer transcriptional repression and increased period length. However, while
CKI has been linked to modulating PER stability and nuclear entry, it remained
unclear which perturbation dominates the period effect, and whether these
processes are sufficient to separate the effects of CKI and FBXL3.

To generate predictions that are consistent across slight differences in model
assumptions, we chose three mathematical models from the literature based on
their moderate size and similar scope \cite{Hirota2012, Leloup2003,
Relogio2011}. The models included, at a minimum, the expression and nuclear
entry mechanisms of PER and CRY. We considered the formation of the PER-CRY
heterodimer as a key step in nuclear entry, which is supported by the fact that,
to the best of our knowledge, all circadian models that consider both PER and
CRY employ this kinetic assumption \cite{Hirota2012, Relogio2011, Leloup2003,
Forger2003, Mirsky2009}.

Since dynamic models of genetic regulatory networks typically suffer from poor
parameter identifiability \cite{Gunawan2006}, we demonstrate that our
predictions are parameter-independent by employing a bootstrap identifiability
analysis \cite{St.John2013}. As part of the bootstrap method, the models were
re-fit to experimental data \cite{Lee2001} while ensuring appropriate protein
stoichiometry. The state
trajectories of the resulting 2000 parameter sets for each model are shown in
Fig. 2, with reasonable agreement between models and experiment.

A first-order period sensitivity analysis, performed on each of the parameter
sets, identified which parameters associated with PER and CRY protein activity
had the greatest effect on period (Fig. S2). To simplify analysis, we present
only those parameters that are associated with experimentally supported
mechanisms of CKI and FBXL3 in Fig. 3. We first tested parameters associated
with potential FBXL3-CRY activity (Fig. 3A) to evaluate if our method matched
the experimentally verified effect of KL001 \cite{Hirota2012}. Since CRY is the
dominant repressor of CLOCK-BMAL1 \cite{Ye2011}, we attribute degradation rates
of the PER-CRY complex to be representative of CRY clearance rates. We found
that only parameters governing nuclear CRY degradation show a period
lengthening effect upon inhibition, while rates associated with cytoplasmic CRY
degradation show period shortening effects. These results match with our
previous assertion that period lengthening occurs via nuclear CRY stabilization
\cite{Hirota2012}. Experimental evidence has also indicated cytoplasmic CRY
stabilization may lead to period shortening \cite{Kurabayashi2010}, a result
consistent with our mathematical results.

We next describe parameters potentially associated with CKI-dependent
regulation of PER localization and stability (Fig. 3B). Since PER is
rate-limiting in the formation of the PER-CRY complex \cite{Lee2001}, rates
associated with complex formation or nuclear import were included in this
analysis. Conversely, we did not include degradation rates of PER-CRY nuclear
repressive complex, since CRY alone is considered the main repressor. While it
was hypothesized in models where PER acts as a direct repressor that the
regulation of PER stability would play the dominant role determining the period
\cite{Gallego2006, Vanselow2006}, our new assumptions revealed that parameters
governing PER degradation showed only non-identifiable responses.  However,
inhibition of rates associated with the nuclear entry of the PER-CRY complex
showed strong period lengthening effects. These results indicate
that under our current understanding of clock kinetics, the regulation of
nuclear import likely plays the prominent role in CKI-dependent period
regulation.

\subsection{Mathematical insights into the different and independent mechanisms
of PER and CRY regulation}
Using the model and parameter set of Hirota {\it et al.}, 2012
\cite{Hirota2012} and the perturbations identified in Fig. 3, we first
confirmed that inhibition of nuclear CRY degradation (vdCn) and PER-CRY nuclear
import (vaC1P) reproduced the experimental period and amplitude effects of the
small molecules KL001 and longdaysin, respectively (Fig.  4A, compare with Fig.
1C). Comparison of the oscillatory profiles of {\it Per} mRNA and nuclear CRY
protein (Fig. 4B) revealed that inhibition of FBXL3-dependent CRY degradation
caused lingering nuclear CRY to not be completely purged each cycle.  This
excess repressor during the accumulating phase of {\it Per} and {\it Cry}
transcripts resulted in lower E box amplitudes, providing a likely explanation
for the effect of KL001.

In contrast, stabilization of cytoplasmic PER (lowering vdP) resulted in reduced
transcriptional amplitude with minimal period effect (Fig. S3), consistent with
experimental findings from the knockdown of $\beta$-TrCP, an F box protein
responsible for PER degradation \cite{Ohsaki2008}. However, other experimental
results have shown that down-regulation of $\beta$-TrCP leads to longer periods
\cite{Reischl2007}, suggesting that further modeling and experimental inquiry is
needed on the role of $\beta$-TrCP in clock regulation. This period lengthening
might be explained through $\beta$-TrCP-mediated stabilization of nuclear
PER-CRY or by using alternative kinetic assumptions for the rate of PER-CRY
binding. 

We further compared the effect of inhibiting PER degradation with inhibiting
nuclear import on the oscillatory profile of key clock proteins (Fig. 4C) to
identify mechanistic differences between the two potential effects of CKI
inhibition. Both perturbations increased cytoplasmic PER, suggesting the two
mechanisms are difficult to distinguish experimentally. Direct stabilization of
PER in the cytoplasm (lowering vdP) lead to two simultaneous trends which shift
the period in opposite directions: it shortened the time delay between
transcription and inactivation by accelerating the accumulation of cytoplasmic
PER and nuclear PER-CRY; and lengthened the repressive phase by increasing the
total amount of PER-CRY which enters the nucleus. These perturbations sped and
slowed the clock, respectively, and resulted in little period change. In
contrast, inhibiting PER-CRY nuclear entry (lowering vaC1P) caused additional
free protein to build in the cytoplasm, delaying nuclear accumulation and
ultimately increasing the total amount of nuclear PER-CRY. Since both of these
trends work to increase period length, inhibiting PER-CRY nuclear entry
resulted in significantly longer cycles. Additionally, the longer cytoplasmic
time delay resulted in increased transcription, yielding slightly higher
amplitudes (Fig. 4B) that closely match the experimental results of the small
molecule longdaysin.

Since CKI likely regulates both stability and subcellular localization of PER
{\it in vivo}, we considered the effects of simultaneously lowering both PER
cytoplasmic degradation and nuclear entry rates (Fig. 5A).  The loss of
oscillations under extreme reduction of both parameters (Fig. 5A, shaded
regions) highlights an interesting role of CKI in conferring robustness to the
circadian clock: since oscillations are lost when import of the PER-CRY complex
to the nucleus ceases to be rhythmic, CKI ensures lingering PER is purged from
the cytoplasm by one pathway or another before E box transcription resumes.
This importance has been proven experimentally, as disruption of CKI-mediated
regulation leads to compromised circadian oscillations \cite{Lee2009a}.

Together, inhibition of CKI by longdaysin may increase the time required
before PER-CRY can enter the nucleus to repress transcription, leading to a
higher amplitude and longer period. In contrast, KL001 lengthens the period by
stabilizing nuclear CRY, resulting in a longer time delay before transcription
resumes and lower amplitude from increased E box repression. PER regulation
through CKI is therefore partitioned to the accumulating phase, controlling the
speed and amount of PER-CRY complex that enters the nucleus. CRY regulation
through FBXL3 is partitioned independently to the repressive phase, controlling
the length of time until CLOCK-BMAL1-dependent transcription resumes (Fig. 6). This
independence was reproduced {\it in silico} by the simultaneous reduction of
nuclear CRY degradation and PER-CRY nuclear import (Fig. 5B), where nonlinear
interactions in amplitude and period between the two perturbations are all but
absent.

\subsection{Conclusion}
An understanding of the interactions between post-translational regulators is
crucial for the further development of circadian pharmacological reagents, as
efficient modulation of clock function will assuredly come from simultaneous
perturbations to many connected species. In this study, we used circadian
reporter cells together with mathematical modeling to provide mechanistic
insight into the differences of CKI- and FBXL3- mediated post-translational
regulation of PER and CRY. As a result, we clarified a process by which CKI
exerts control over the circadian period, demonstrated through both the
hyperphosphorylating CKI$\epsilon^\mathrm{tau}$ mutant and small molecule CKI
inhibitors, such as longdaysin. In developing our predictions, we have used
multiple models and parameterizations to ensure our mechanisms are consistent
across many {\it in silico} realizations. These results reinforce the notion
that computational modeling is essential in interpreting results in systems
with complicated oscillatory feedback. Additionally, {\it in silico} analyses
reveal hidden design principles of biological networks, as this work highlights
the importance of the CKI family of kinases in conferring robustness to the
circadian cycle. 
 

