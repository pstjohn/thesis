{
\singlespace
\setlength{\parindent}{0em}
\setlength{\parskip}{1em}

\chapter*{Vita of Peter C. St. John}
\markboth{Vita of Peter C. St. John}{Vita of Peter C. St. John}

\subsection*{Contact Information}
\vspace{.05in}
\begin{tabular}{@{}p{3.5in}p{4in}}
Department of Chemical Engineering   & {\it Phone:}  (508) 494-2474\\             
University of California, Santa Barbara & {\it E-mail:} pstjohn@engineering.ucsb.edu\\       
Santa Barbara, CA 93106-5080 & {\it Office:} Engineering II, Rm. 1508\\
\end{tabular}


\subsection*{Education}
  {\bf University of California, Santa Barbara} \hfill {\bf September 2010 - present} \\
  {\em Ph.D. Candidate, Department of Chemical Engineering} \hfill Santa Barbara,
  California \\[-3ex]
  % \begin{itemize}
  % \item Advisor:  Francis J. Doyle III
  % \item GPA: 3.68
  % \end{itemize}

  {\bf Tufts University}  \hfill {\bf September 2006 - May 2010}\\
  {\em BS, Chemical and Biological Engineering} \hfill Medford,
  Massachusetts\\[-3ex]
  % \begin{itemize}
  % \item Summa Cum Laude, Thesis Honors, Dean's List every semester
  % \item GPA: 3.79
  % \end{itemize}

\subsection*{Honors and Awards} 
CAST Student Travel Grant \hfill {\bfseries September 2014}\\
Society for Research on Biological Rhythms (SRBR) Research Merit Award \hfill {\bf June 2014}\\
Best Poster, Center for Chronobiology Symposium, UCSD \hfill {\bf February 2014} \\
$1^\textrm{st}$ Place, SRBR Logo Competition \hfill {\bf January 2014} \\
Mitsubishi Chemical Fellowship Recipient \hfill {\bf 2012-2015} \\
UCSB Scienceline 2011-2012 Life Science Outstanding Answerer \hfill {\bf June 2012} \\
National Science Foundation GRFP Honorable Mention \hfill {\bf April 2011} \\
Class of 1947 Victor Prather Prize \hfill {\bf May 2010}\\
Max Tischler Prize Scholarship \hfill  {\bf May 2009}\\
Elected to Tau Beta Pi \hfill  {\bf September 2008}\\

\subsection*{Publications}
{\bfseries St.\ John, P.C.,} Taylor, S.R., Abel, J.H., and F.J. Doyle III. Amplitude metrics for cellular circadian bioluminescence reporters (2014) {\itshape Biophysical Journal}, 107 (11) pp. 2712-2722

{\bfseries St.\ John, P.C.,} Hirota, T., Kay, S.A. and F.J. Doyle III.
Spatiotemporal separation of PER and CRY posttranslational regulation in the
mammalian circadian clock (2014) {\itshape PNAS}, 111 (5) pp. 2040-2045.

Yang, R., Rodriguez-Fernandez, M., {\bfseries St.\ John, P.C.,} and F.J. Doyle III. Chapter 8 -- Systems Biology (2014) In E. Carson and C. Cobelli (Eds.)
{\itshape Modelling Methodology for Physiology and Medicine, 2nd Edition}, pp.
159-187.

{\bfseries St.\ John, P.C.,} and F.J. Doyle III. Estimating confidence intervals in predicted responses for oscillatory biological models (2013) {\itshape BMC Systems Biology} 7:71.
 
Hirota, T., Lee, J.W., {\bfseries St.\ John, P.C.}, Sawa, M., Iwaisako, K.,
Noguchi, T., Pongsawakul, P.Y., Sonntag, T., Welsh, D.K., Brenner, D.A., Doyle,
F.J. III, Schultz, P.G., Kay, S.A.,  Identification of small molecule
activators of cryptochrome (2012) {\itshape Science}, 337 (6098) pp. 1094-1097.

Murphy A.R., {\bf St.\ John P.C.}, Kaplan D.L. Modification of silk fibroin using diazonium coupling chemistry and the effects on hMSC proliferation and differentiation (2008) {\em Biomaterials}, 29 (19), pp. 2829-2838.  

\subsection*{Contributed Talks}
{\bfseries St.\ John, P.C.,} and F.J. Doyle III. November 2014. Development of Amplitude Response Curves for Single-Cell and Population-Level Circadian Systems. {\itshape To be presented} at the 2014 AIChE Annual Meeting, Atlanta, GA

% {\bfseries St.\ John, P.C.} October 2014. \textit{Mitsubishi Lecture:} Mathematical approaches to understanding amplitude in mammalian circadian rhythms. Presented at the $7^{\mathrm{th}}$ Annual Clorox-Amgen Graduate Student Symposium, Santa Barbara, CA.

{\bfseries St.\ John, P.C.,} and F.J. Doyle III. June 2014. Amplitude metrics for uncoupled cellular circadian bioluminescence reporters. Presented at the Society for Research on Biological Rhythms Meeting, Big Sky, MT.

{\bfseries St.\ John, P.C.,} and F.J. Doyle III. October 2012. Cryptochrome balancing for period control: mathematical insights into circadian clock design. Presented at the Model-based Analysis and Control of Cellular Processes Workshop, Purdue University, West Lafayette, IN.

\subsection*{Poster Presentations}
{\bfseries St.\ John, P.C.,} and F.J. Doyle III. February 2014. Spatiotemporal
separation of PER and CRY post-translation regulation. Presented at the UCSD
Center for Chronobiology Symposium, La Jolla, CA.

{\bfseries St.\ John, P.C.,} T. Hirota, S.A. Kay, and F.J. Doyle III. July 2013. Estimating confidence intervals in model predictions to determine plausible mechanisms for small molecule modifiers. Presented at the Chronobiology Gordon Research Conference, Newport, RI.

{\bfseries St.\ John, P.C.,} and F.J. Doyle III. February 2013. Predictive confidence intervals from mathematical circadian models. Presented at the UCSD Center for Chronobiology Symposium, La Jolla, CA.

% {\bfseries St.\ John, P.C.,} and F.J. Doyle III. October 2012. Modeling of circadian rhythms for small molecule modulator characterization. Presented at the Amgen-Clorox Graduate Student Symposium, Santa Barbara, CA.

{\bfseries St.\ John, P.C.,} T. Hirota, S. Kay, and F.J. Doyle III. May 2012. Cryptochrome balancing for period control. Presented at the Meeting of the Society for Research on Biological Rhythms, Destin, FL.

{\bfseries St.\ John, P.C.,} and F.J. Doyle III. February 2012. Perturbation analysis of circadian clock degradation. Presented at the UCSD Center for Chronobiology Symposium, La Jolla, CA.

{\bfseries St.\ John, P.C.,} M. Rodriguez-Fernandez, and F.J. Doyle III.  June 2011. Advanced global optimization and sensitivity techniques for analyzing deterministic circadian models. Presented at the 12th Annual UC Systemwide Bioengineering Symposium, Santa Barbara, CA.

% {\bfseries St.\ John, P.C.,} and F.J. Doyle III. September 2011. Perturbation analysis of circadian clock function. Presented at the Amgen-Clorox Graduate Student Symposium, Santa Barbara, CA.

% {\bfseries St.\ John, P.C.,} J. A. Hanson, and T. J. Deming. August 2009. Release from Nanoscale Water-in-Oil-in-Water Double Emulsions Stabilized by Block Copolypeptide Surfactants. Presented at the UCLA NanoCER Poster Session, Los Angeles, CA.

% {\bfseries St.\ John, P.C.,} and H. Yi. April 2008. Catalytic hydrodechlorination of 2-chlorophenol using viral-templated palladium nanoparticles. Presented at the 10th Annual Undergraduate Research and Scholarship Symposium, Medford MA.


\subsection*{Seminars}
{\bfseries St.\ John, P.C.} May 2013. Sensitivity analysis in the study of
mammalian circadian rhythms. Presented at the UCSB NSF-IGERT systems biology
seminar series, Santa Barbara, CA.

\subsection*{Research Experience}

{\bf University of California, Santa Barbara} \hfill {\bf September 2010 - present}\\
{\em Ph.D. Candidate} \hfill Santa Barbara, California\\
Computational analysis of the mammalian circadian clock, with a focus on
elucidating the functional design consequences of the underlying genetic
regulatory network.\\
% This work deals primarily with simulating biological
% networks as systems of coupled, nonlinear, deterministic ordinary differential
% equations.\\
Advisor: Francis J. Doyle III \\
Department: Chemical Engineering

{\bf Tufts University} \hfill {\bf September 2009 - May 2010}\\
{\em Senior Honors Thesis} \hfill Medford, Massachusetts\\
Catalytic Hydrodechlorination of 2-Chlorophenol using Viral-templated Palladium Nanoparticles\\
Advisor: Hyunmin Yi \\
Department: Chemical and Biological Engineering

{\bf University of California, Los Angeles} \hfill {\bf June 2009 - August 2009}\\
{\em UCLA NanoCER Program (NSF REU)} \hfill Los Angeles, California\\
Encapsulation Efficiencies and Release Rates from Water-in-Oil-in-Water Nanoemulsions\\
Advisor: Timothy Deming\\
Department: Bioengineering

{\bf Tufts University} \hfill {\bf June 2008 - August 2008}\\
{\em Tufts Summer Scholars} \hfill Medford, Massachusetts\\
Hydrodechlorination of 2-Chlorophenol with Palladium Nanoparticles on Genetically Modified Tobacco Mosaic Virus Scaffolds\\
Advisor: Hyunmin Yi\\
Department: Chemical and Biological Engineering

{\bf Tufts University} \hfill {\bf October 2007 - May 2008}\\
{\em Undergraduate Research Credit} \hfill Medford, Massachusetts\\
Chemically Modified Silk Fibroin Based Scaffolds for Bone Tissue Engineering\\
Advisor: David Kaplan\\
Department: Biomedical Engineering

\subsection*{Teaching Experience}
%132c again
{\bf University of California, Santa Barbara}  \hfill {\bf January 2013 -
  December 2013}\\
{\em Teaching Assistant, ChE132c}\hfill  Santa Barbara, California\\
Helped teach undergraduate statistics for two subsequent years: gave three
lectures, held office hours and review sessions, and graded homeworks.

{\bf University of California, Santa Barbara} \hfill {\bf March 2012 - June 2012}\\
{\em Teaching Assistant, ChE180a}\hfill  Santa Barbara, California\\
Designed and ran experiments for the junior laboratory course. Also helped in grading student reports.

\subsection*{Community Involvement}
{\bf Peer Review \hfill January 2014-present}\\
Reviewer for Biophysical Journal;  IEEE Control Systems Society Conference; 21st International Symposium on
Mathematical Theory of Networks and Systems

{\bf Scienceline ``Ask A Scientist"} \hfill {\bf December 2010 - present}\\
Answers science and engineering questions posed by students and teachers from local K-12 schools. \\
{\bfseries Website:} www.scienceline.ucsb.edu

{\bf UCSB Discover Engineering Weekend} \hfill {\bf May 2011}\\
Helped organize and run a weekend for local high school students to learn basic engineering principles and apply their knowledge to build miniature alternative energy cars.

% \subsection*{Leadership Experiences}
% {\bf Tufts Chapter of the AIChE}\\
% {\em Vice President} \hfill {\bf September 2008 - May 2010}\\
% (American Institute of Chemical Engineers) Coordinated department-wide social, professional, and community service events

\subsection*{Mentoring Experience}
Amanda Luan, Undergraduate Student, ICB SSB URAP \hfill {\bf June 2014 - December 2014} \\
Lukas Widmer, Masters Student, ETH Zurich \hfill {\bf April 2012 - February 2013} \\
Andrew Barisser, Rotation Student, BMSE UCSB \hfill {\bf September 2012 - December 2012}

}
